\title{Module 5 Dynamic Programming Assignment}
\author{
    Tyler Kim\\
    tkj9ep
}

\documentclass[12pt, a4paper]{article}
\usepackage{amssymb, amsthm, amssymb}
\begin{document}
\maketitle
\section*{\centering Response 1}
\textbf{Given}: \(n\) for the total number for doors and \(S\) for number or secured doors.
\\
\\
My algorithm would initialize an array of integers \(A\) of size \(n + 1\) and fill the first \(S-1\) indices to 0 and \(A[S]\) to 1. 
The algorithm will iterate through all the indices of \(A\) and update each element of the array such that index \(A[i] = A[i - 1] + 2^{i - S}\).
The algorithm will return \(A[n]\).

\newpage
\section*{\centering Response 2}

\newpage
\section*{\centering Response 3}

\textbf{Given}: \(w\) = max size of the tile cover
\\
\\
The algorithm will first create an array \(A\) of integers of size \(w + 1\) and assign \(A[0] = 0, A[1] = 1,\) and \(A[2] = 2\).
The algorithm will iterate through each of the cells after \(A[2]\) and update each cell such that \(A[i] = A[i - 1] + A[i - 2]\).
Once the algorithm fully fills all the values in \(A\), it will return the output of the value in the very last cell.
The runtime of the algorithm is linear or \(\Theta(w)\) because the algorithm has to loop through all \(w\) values.   

\newpage
\section*{\centering Response 4}
j

\end{document} 