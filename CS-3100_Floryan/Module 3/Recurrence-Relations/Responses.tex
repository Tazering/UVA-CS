\title{Module 3 Recurrence Relations Assignment}
\author{
    Tyler Kim \\
    tkj9ep
}
\date{October 11, 2022}

\documentclass[12pt, a4paper]{article}
\usepackage{amssymb, amsthm, amssymb}
\begin{document}
\newenvironment{claim}[1]{\par\noindent\underline{Claim:}\space#1}{}
\newenvironment{basecase}[1]{\par\noindent\underline{Base Case:}\space#1}{}
\newenvironment{inductivehypothesis}[1]{\par\noindent\underline{Inductive Hypothesis:}\space#1}{}
\newenvironment{inductive}[1]{\par\noindent\underline{Inductive Case:}\space#1}{}
\newenvironment{guess}[1]{\par\noindent{Guess:}\space#1}{}
\newenvironment{prove}[1]{\par\noindent{Prove:}\space#1}{}
\maketitle
\section*{\centering Response 1}
    1. The algorithm would take in two integer inputs, \textit{int start} and \textit{int end}, and would return an integer.
    The algorithm would check if \(end - start == 2\), if so the algorithm would call f([0], [1]).
    If f([0], [1]) == -1, then the algorithm returns the parameter \textit{start} and if f([0],[1]) == 1, then the algorithm returns the parameter \textit{end}.
    Else, the algorithm would calculate mid by taking the size, \textit{n}, and dividing it by 2. 
    If \textit{n} is even, the algorithm would check if the left or right side of the array is bigger by calling f([0..mid], [mid..end]).
    If f([0..mid], [mid..end]) == -1, then the algorithm would recursively call on itself with \textit{start} = 0 and \textit{end} = mid.
    If the function resulted in 1, then the algorithm will call on [mid..end] of array instead.
    If \textit{n} is odd, then the function would check if the left half and right half (excluding the exact middle value) of the array is bigger or not.
    If calling f([0..mid],[mid+1..end]) is 0, then \textit{mid} must be the index. If not, then the algorithm will proceed the recursion calls like previously described.\\
    2. The recurrence relation is \(T(n) = T(\frac{n}{2}) + 2f()\) because the algorithm makes one recursive call and calls f() twice at most.\\
    \\
    3. \textbf{Given}: \(T(n) = T(\frac{n}{2}) + 2f()\)\\
    \\
    \textbf{Use Master's Theorem}\\

    Let \(a = 1, b = 2, k = \log_{2}(1) = 0, f(n) = 2f() = 2n\)\\
    Since \(n^{k} = n^{0} = 1 < 2n\), Case 3 applies.\\ 
    Need to check regularity: \(af(\frac{n}{b}) \leq cf(n)\). 
    This means that \(c < \frac{af(\frac{n}{b})}{f(n)} < 1\).
    This means that \(c < \frac{\frac{2n}{2}}{2n} = \frac{1}{2} \) which proves the regularity theorem; thus, case 3 of the Master's Theorem can be used.\\
    \(T(n) \in \Theta(f(n))\)\\
    \(\therefore \Theta()\) 
    \newpage
    
\section*{\centering Response 2}
    \textbf{Given}: \(T(n) = T(n-1) + n\)\\
    \\
    \textbf{Unroll the Recurrence}\\
    Let \textit{d} denote level of unrolling\\
    \\
    \textit{d} = 0: \(T(n) = T(n-1) + n\)\\
    \textit{d} = 1: \(T(n) = [T(n-2) + (n-1)] + n = T(n-2) + 2n - 1\)\\
    \textit{d} = 2: \(T(n) = [T(n-3) + (n-2)] + 2n - 1 = T(n-3) + 3n - 3\)\\
    \textit{d} = 3: \(T(n) = [T(n-4) + (n-4)] + 3n - 3 = T(n-4) + 4n - 7\)\\
    \\
    General Pattern: \(T(n) = T(n - (d + 1)) + (d - 1)n - \frac{d(d+1)}{2}\)\\
    \\
    The base case when \(T(1)\) is reached when \(n-d - 1 = 1\).\\
    Solve for \(d\): \\
    \(n - d = 2\)\\
    \(-d = 2-n\)\\
    \(d=n-2\) \\
    \\
    Plug \(d\) back in:\\
    \(T(n)=T(n-((n-2)+1))+((n-2)-1)n-\frac{(n-2)(n - 2 + 1)}{2}\)\\
    \(T(n)=T(1)+(n-3)n-\frac{n^2-2n+2}{2}\)\\
    \(T(n)=n^2-3n-\frac{n^2-2n+2}{2} = \Theta{(n^2)}\)\\
    \(\therefore \Theta{(n^2)}\)
    \newpage

\section*{\centering Response 3}
    \begin{proof}
        \begin{claim}
            \(T(n) = 4T(\frac{n}{3} + n \in O(n^{\log_{3}(4)})\)\\
        \end{claim}
        \begin{guess}
            \(O(n^{\log_{3}(4)})\)
        \end{guess}\\
        \begin{prove}
            \(T(n) \leq cn^{\log_{3}(4)}\) where \(c\) is a constant. 
        \end{prove}\\
        \begin{basecase}
            \(n = 3\)\\
            \\
            \(T(3) \le c \cdot 3^{\log_{3}(4)}\)\\
            \(T(3) \le c \cdot 4\)\\
            \(4T(\frac{3}{3}) + 3 \leq 4c\)\\
            \(4T(1) + 3 \leq 4c\)\\
            \(4 \cdot 1 + 3 \leq 4c\)\\
            \(7 \leq 4c\) when \(c \geq \frac{7}{4}\)\\
            $\therefore$ Since \(3 \geq \frac{7}{4}\), the base case holds.\\
        \end{basecase}
        \begin{inductivehypothesis}
            Let \(k \leq n\) such that \(T(k) \leq c \cdot n^{\log_{3}(4)} - dk\) where \(d\) is a constant.\\
        \end{inductivehypothesis}
        \begin{inductive}\\
            \(T(n) = 4T(\frac{n}{3}) + n\)\\
            \(T(n) \leq 4[c \cdot (\frac{n}{3})^{\log_{3}(4)} - dn] + \frac{n}{3}\)\\
            \(T(n) \leq 4[c \cdot \frac{n^{\log_{3}(4)}}{3^{\log_{3}(4)}} - dn] + \frac{n}{3}\)\\
            \(T(n) \leq 4[c \cdot \frac{n^{\log_{3}(4)}}{4} - d] + \frac{n}{3}\)\\
            \(T(n) \leq cn^{\log_{3}(4)} - 4dn + \frac{n}{3}\)\\
            Since \(4dn\) is larger than \(\frac{n}{3}\), we can transform the recurrence relation to \(T(n) \leq cn^{\log_{3}(4)}-dn\)
        \end{inductive}\\
        $\therefore$ the inequality was proven and the claim is true.

    \end{proof}

\section*{\centering Response 4}
    \textbf{Given}: \(T(n) = 2T(\frac{n}{4}) + 1\)\\
    \\
    Apply Master Theorem:\\
    A = 2, B = 4, \(f(n) = 1\)\\
    \(k = \frac{\log{2}}{\log{4}} = \frac{1}{2}\)\\
    Compare \(f(n) = 1\) to \(n^{\frac{1}{2}}\)\\
    Since \(f(n) = O(n^{\frac{1}{2} - \epsilon})\) where \(\epsilon = \frac{1}{2}\), Case 1 applies:\\
    \(T(n) \in \Theta{(n^{\frac{1}{2}})}\)\\
    \(\therefore \Theta(n^{\frac{1}{2}})\)


\section*{\centering Response 5}
    \textbf{Given}: \(T(n) = 2T(\frac{n}{4}) + \sqrt{n}\)\\
    \\
    Apply Master Theorem:\\
    A = 2, B = 4, \(f(n) = \sqrt{n}\)\\
    \(k = \frac{\log{2}}{\log{4}} = \frac{1}{2}\)\\
    Compare \(f(n) = \sqrt{n}\) to \(n^{\frac{1}{2}}\)\\
    Since \(f(n) = \sqrt{n}\) is equal to \(n^k = n^{\frac{1}{2}}\), then we apply Case 2:\\
    \(T(n) = \Theta{f(n)log(n)} = \Theta{(n^{\frac{1}{2}}log(n^{\frac{1}{2}}))}\)\\
    $\therefore$ \(\Theta{(nlog(n))}\)

\section*{\centering Response 6}
    \textbf{Given}: \(T(n)=2T(\frac{n}{2}) + n\)\\
    \\
    Apply Master Theorem:\\
    A = 2, B = 4, \(f(n) = n\)\\
    \(k = \frac{\log{2}}{\log{4}} = \frac{1}{2}\)\\
    Compare \(f(n) = n\) to \(n^{\frac{1}{2}}\)\\
    Since \(n^{\frac{1}{2} - \epsilon}\) results in $\epsilon$ = $\frac{1}{2}$, and \(cf(n) \geq n^{\frac{1}{2}}\), apply Case 3:\\
    \(T(n) \in \Theta{(f(n))}\).\\
    $\therefore$ $\Theta{(n)}$\\

\section*{\centering Response 7}
    \textbf{Given}: \(T(n)=2T(\frac{n}{4})+n^2\)\\
    \\
    Apply Master Theorem:\\
    A = 2, B = 4, \(f(n) = n^2\)\\
    \(k = \frac{\log{2}}{\log{4}} = \frac{1}{2}\)\\
    Compare \(f(n) = n^2\) and \(n^{\frac{1}{2}}\)\\
    Since \(n^{\frac{1}{2}+\epsilon}\) results in $\epsilon$ = 1.5 and \(cf(n) \geq n^{\frac{1}{2}}\), apply Case 3:\\
    \(T(n) \in \Theta{f(n)}\).\\
    $\therefore$ $\Theta{(f(n))}$

\end{document}