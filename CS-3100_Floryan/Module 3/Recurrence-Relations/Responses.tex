\title{Module 3 Recurrence Relations Assignment}
\author{
    Tyler Kim \\
    tkj9ep
}
\date{October 11, 2022}

\documentclass[12pt, a4paper]{article}
\usepackage{amssymb, amsthm, amssymb}
\begin{document}
\maketitle
\section*{\centering Response 1}

\section*{\centering Response 2}
    \textbf{Given}: \(T(n) = T(n-1) + n\)\\
    \\
    \textbf{Unroll the Recurrence}\\
    Let \textit{d} denote level of unrolling\\
    \\
    \textit{d} = 1: \(T(n) = T(n-1) + n\)\\
    \textit{d} = 2: \(T(n) = [T(n-2) + (n-1)] + n = T(n-2) + 2n - 1\)\\
    \textit{d} = 3: \(T(n) = [T(n-3) + (n-2)] + 2n - 1 = T(n-3) + 3n - 3\)\\
    \textit{d} = 4: \(T(n) = [T(n-4) + (n-4)] + 3n - 3 = T(n-4) + 4n - 7\)\\
    \\
    General Pattern: \(T(n) = T(n-d) + dn - (2^{d-1} - 1)\)\\
    \\
    The base case when \(T(1)\) is reached when \(n-d = 1\).\\
    Solve for \(d\): \\
    \(n - d = 1\)\\
    \(-d = 1-n\)\\
    \(d=n-1\) \\
    \\
    Plug \(d\) back in:\\
    \(T(n)=T(n-(n-1))+(n-1)n-(2^{n-1-1}-1)\)\\
    \(T(n)=T(1)+n^2-n-2^{n-1}+1\)\\
    \(T(n)=n^2-n-2^{n-2}+1 = \Theta{(2^n)}\)\\
    \(\therefore \Theta{(2^n)}\)
    \newpage

\section*{\centering Response 3}

\section*{\centering Response 4}
    \textbf{Given}: \(T(n) = 2T(\frac{n}{4}) + 1\)\\
    \\
    Apply Master Theorem:\\
    A = 2, B = 4, \(f(n) = 1\)\\
    \(k = \frac{\log{2}}{\log{4}} = \frac{1}{2}\)\\
    Compare \(f(n) = 1\) to \(n^{\frac{1}{2}}\)\\
    Since \(f(n) = O(n^{\frac{1}{2} - \epsilon})\) where \(\epsilon = \frac{1}{2}\), Case 1 applies:\\
    \(T(n) \in \Theta{(n^{\frac{1}{2}})}\)\\
    The solution must be \(T(n) = \Theta{(n)}\) since \(k = \frac{1}{2}\) and rounds up to 1


\section*{\centering Response 5}
    \textbf{Given}: \(T(n) = 2T(\frac{n}{4}) + \sqrt{n}\)\\
    \\
    Apply Master Theorem:\\
    A = 2, B = 4, \(f(n) = \sqrt{n}\)\\
    \(k = \frac{\log{2}}{\log{4}} = \frac{1}{2}\)\\
    Compare \(f(n) = \sqrt{n}\) to \(n^{\frac{1}{2}}\)\\
    Since \(f(n) = \sqrt{n}\) is equal to \(n^k = n^{\frac{1}{2}}\), then we apply Case 2:\\
    \(T(n) = \Theta{f(n)log(n)} = \Theta{(n^{\frac{1}{2}}log(n^{\frac{1}{2}}))}\)\\
    $\therefore$ \(\Theta{(nlog(n))}\)

\section*{\centering Response 6}
    \textbf{Given}: \(T(n)=2T(\frac{n}{2}) + n\)\\
    \\
    Apply Master Theorem:\\
    A = 2, B = 4, \(f(n) = n\)\\
    \(k = \frac{\log{2}}{\log{4}} = \frac{1}{2}\)\\
    Compare \(f(n) = n\) to \(n^{\frac{1}{2}}\)\\
    Since \(n^{\frac{1}{2} - \epsilon}\) results in $\epsilon$ = $\frac{1}{2}$, and \(cf(n) \geq n^{\frac{1}{2}}\), apply Case 3:\\
    \(T(n) \in \Theta{(f(n))}\).\\
    $\therefore$ $\Theta{(n)}$\\

\section*{\centering Response 7}
    \textbf{Given}: \(T(n)=2T(\frac{n}{4})+n^2\)\\
    \\
    Apply Master Theorem:\\
    A = 2, B = 4, \(f(n) = n^2\)\\
    \(k = \frac{\log{2}}{\log{4}} = \frac{1}{2}\)\\
    Compare \(f(n) = n^2\) and \(n^{\frac{1}{2}}\)\\
    Since \(n^{\frac{1}{2}+\epsilon}\) results in $\epsilon$ = 1.5 and \(cf(n) \geq n^{\frac{1}{2}}\), apply Case 3:\\
    \(T(n) \in \Theta{f(n)}\).\\
    $\therefore$ $\Theta{(f(n))}$

\end{document}